\begin{abstract}

Physicists use accelerators such as the Large Hadron Collider (LHC) to produce new particles in order to understand the elementary structure of matter. These colliders produce different types of particles which are identical in many aspects but differ in their kinematic properties. In this project, we attempt to distinguish between those collisions that produce a particle of interest (signal) and those that produce other particles (background). This task is considered to be very difficult as conventional wisdom dictates the need for using complex parameters that are not directly observable for classification. To solve this problem, we explore a number of different techniques and evaluate these techniques in terms of accuracy and throughput. \emph{Our best classifiers outperform the state-of-the-art implementations~\cite{DeepNN} by over 2\% in terms of accuracy}. We also provide a qualitative discussion on the possible trade-offs for different techniques.


\end{abstract}
