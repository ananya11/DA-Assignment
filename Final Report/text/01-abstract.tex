\begin{abstract}

Physicists use high-energy particle colliders such as the Large Hadron Collider (LHC) to produce new particles in order to understand the elementary structure of matter. These colliders produce different types of particles which are identical in many aspects but differ in their kinematic features. In this project, we attempt to distinguish between collisions that produce a particle of interest (signal) and those that produce other particles (background). This task is considered to be very difficult as conventional wisdom dictates the need for using complex parameters that are not directly observable for classification. We explore a number of different techniques and evaluate them in terms of accuracy and throughput. Our best classifiers outperform the state-of-the-art by over 2\% in terms of accuracy. We also analyze the trade-offs between several techniques explored.


\end{abstract}
