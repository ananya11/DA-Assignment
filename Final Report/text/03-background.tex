\section{Background}
\label{sec:background}

The LHC collides bunches of proton every 50 ns producing a random number of proton-proton collisions called events. These events produce new particles, most of which are very unstable and decays quickly. The ATLAS detector measures three properties of these surviving particles: type, energy and 3D direction of the particle. Based on these properties,  the property of the heaviest primary particles is inferred. An online trigger system selects about 400 events per second producing 3 pb of raw data per year.

Each event contains about ten particles of interest in the final state. The particles of interest for this challenge are electrons, muons, hadronic tau, jets and missing transverse energy. Electrons and muons live longer enough to reach the detector, so their properties (direction and energy) can be measured directly. The other particles such as tau, hadron and jets decays into different particles and hence their properties are inferred using the law of momentum conservation.  The measured momenta of all the particles of the event is the primary information provided in the dataset. 

Higgs boson is a particle which is responsible for the mass of the other elementary particles. In the original discovery, the Higgs boson was seen decaying into γγ, WW, and ZZ, which are all boson pairs~\cite{Ananya1,Ananya2} . The signal class comprises of events in which Higgs boson decays into two taus. Our goal is to identify those signals from a significantly higher number of background events.
