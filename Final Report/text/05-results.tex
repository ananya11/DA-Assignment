\section{Results}
\label{sec:results}

In this section, we describe the dataset used, hardware details, accuracy, and throughput results for the different classifiers.

\subsection{Dataset}

We make use of the dataset provided by Kaggle~\cite{Kaggle-Dataset}. This is the cleaned up data from the original dataset provided by physicists at the UC Irvine Machine Learning Repository. This data set contains \emph{250000} instances. Each instance (row) in the dataset describes a collision event detected by the collider. Events are described by the kinematic properties (such as direction and momentum) of the particles produced in a collision. A set of 17 features describe these kinematic properties. In addition, 13 derived features that the physicists deemed important are also included in the dataset. \emph{200000} instances from the original dataset is used for \emph{training} and the remaining \emph{50000} instances is used for \emph{testing}.

\subsection{Hardware details}

All experiments were run on a MacBook Pro using a \emph{4-core} Intel Core i7 processor running at 2.5\,GHz. This machine has \emph{256\,KB} of L2 cache per core, \emph{6\,MB} of L3 cache, and \emph{16\,GB} of DDR3-1600\,MHz memory. Modeling and prediction overheads of all techniques were measured and this machine and the corresponding throughput results presented in subsequent sections.

\subsection{Accuracy Results}

\subsubsection{Bayesian Classifiers}

\subsubsection{Function-based Classifiers}


\subsubsection{Tree-based Classifiers}

\subsubsection{Instance-based Classifiers}

\subsubsection{Neural Network}
As mentioned before we used multiple NN technics to classify the data between a signal and a background, our results were almost similar. Among the 4 neural networks used here, feedforward and cascade forward network has the best ROC curve (AUC) of 0.91, signal precision and recall as 0.79 and 0.74 respectively.  

We noticed that for each network model, the testing performance stops increasing with the increase in the number of neurons in the hidden layers after a certain point . Liu et al.~\cite{NN-Result} call this as the stop criterion. Beyond this point the neural network overestimates the complexity of the target problem which causes overfitting.

Recently there has been substantial interest in feed forward network with many layers. However, we restricted ourselves to only two layers (one hidden and one output layer) as we noticed an increase in overfitting when the number of layers in a feed forward network is greater than 2.  Similarly in the custom network that we designed, the neural network's performance improves when we increase the number of hidden layers from 2 to 3, the AUC is 0.89 but it is still less than other ANN model with 2 layers. 


\subsubsection{Ensemble Methods and Meta-classifiers}

\subsection{Throughput Results}


%\begin{table}[b]
%\centering 
%\caption{Mean error \% for application-independent models} 
%\begin{tabular}{|r|c|c|c|c|} 
%\hline 
%Models & \multicolumn{2}{|c|}{C2075} & \multicolumn{2}{|c|}{K20c}  \\ 
%\cline{2-5} 
%	   & Basic & Temp-aware & Basic & Temp-aware \\ 
%\hline
%SLR 	& 17.96 & 8.59 		& 21.67 & 9.44 	\\ 
%MLR 	& 11.59  & \textbf{4.49}	& 18.66 & 8.29 	\\ 
%MLR+I 	& 14.02 & 6.83 		& 14.74 & \textbf{6.14} 	\\ 
%QMLR 	& 14.83 & 6.42 		& 15.46 & 7.82 	\\ 
%QMLR+I 	& 19.05 & 10.31 		& 19.56 & 8.86 	\\ 
%\hline
%\end{tabular}
%\label{tab:AI-Summary} 
%\end{table}

\subsection{Discussion}

